%\documentclass{beamer}
\documentclass[10pt]{beamer}
% This file is a solution template for:

% - Talk at a conference/colloquium.
% - Talk length is about 20min.
% - Style is ornate.

% Copyright 2004 by Till Tantau <tantau@users.sourceforge.net>.
%
% In principle, this file can be redistributed and/or modified under
% the terms of the GNU Public License, version 2.
%
% However, this file is supposed to be a template to be modified
% for your own needs. For this reason, if you use this file as a
% template and not specifically distribute it as part of a another
% package/program, I grant the extra permission to freely copy and
% modify this file as you see fit and even to delete this copyright
% notice.

% See spinmodel_gamgconfig_LSE.m, spinmodel_gamgconfig_JSE.m, spinmodel_gamgconfig_JSELSE.m
% for the figures.
%

\mode<presentation>
{
% \usetheme{default}
 \usetheme{Warsaw}
% \usetheme{Rochester}
%\usetheme{Madrid}
%\usetheme{Pittsburgh}
%\usetheme{Antibes}
%\usetheme{Montpellier}
%\usetheme{Berkeley}
%\usetheme{PaloAlto}
%\usetheme{Goettingen}
%\usetheme{Marburg}
%\usetheme{Hannover}
%\usetheme{Berlin}
%\usetheme{Ilmenau}
%\usetheme{Dresden}
%\usetheme{Darmstadt}
%\usetheme{Frankfurt}
%\usetheme{Singapore}
%\usetheme{Szeged}
%\usetheme{Copenhagen}
%\usetheme{Malmoe}
% or ...

%\usecolortheme[named=SeaGreen]{structure})
\usecolortheme{crane}
\usepackage[demo]{graphicx}
\usepackage{caption}
\usepackage{subcaption}
%\usecolortheme[named=SeaGreen]{structure}).
\setbeamercovered{transparent}
%or whatever (possibly just delete it)
%\setbeamercolor{frametitle}{fg=blue,bg=cyan}
}
% Predefined colors: red, green, blue, cyan, magenta, yellow, black,
% darkgray, gray, lightgray, orange, violet, purple, and brown
% albatross, beetle, crane, dove, fly, seagull


\usepackage[english]{babel}
% or whatever

\usepackage[latin1]{inputenc}
% or whatever

\usepackage{times}
\usepackage[T1]{fontenc}
% Or whatever. Note that the encoding and the font should match. If T1
% does not look nice, try deleting the line with the fontenc.


\usepackage{graphicx}
\usepackage{multimedia}





\newcommand{\bc}{\begin{center}}
\newcommand{\ec}{\end{center}}
\newcommand{\be}{\begin{equation}}
\newcommand{\ee}{\end{equation}}
\newcommand{\bea}{\begin{eqnarray}}
\newcommand{\eea}{\end{eqnarray}}
\newcommand{\beq}{\begin{eqnarray*}}
\newcommand{\eeq}{\end{eqnarray*}}


\begin{document}

\title[Main title]{main title}
%\title[Market microstructure and high frequency models]{Market microstructure and high frequency models}

\author[] % (optional, use only with lots of authors)
%{More authors} % \\

% - Give the names in the same order as the appear in the paper.
% - Use the \inst{?} command only if the authors have different
%   affiliation.

% \institute[] % (optional, but mostly needed)
 {Project}
 % - Use the \inst command only if there are several affiliations.
% - Keep it simple, no one is interested in your street address.

%\date[July 2013] % (optional, should be abbreviation of conference name)
% - Either use conference name or its abbreviation.
% - Not really informative to the audience, more for people (including
%   yourself) who are reading the slides online

\subject{}
% This is only inserted into the PDF information catalog. Can be left
% out.

% If you have a file called "university-logo-filename.xxx", where xxx
% is a graphic format that can be processed by latex or pdflatex,
% resp., then you can add a logo as follows:

 %\pgfdeclareimage[height=0.5cm]{corporate-logo}{QTLogo.jpg}
% \logo{\pgfuseimage{}}
 %\pgfuseimage{corporate-logo}}

% Delete this, if you do not want the table of contents to pop up at
% the beginning of each subsection:


% If you wish to uncover everything in a step-wise fashion, uncomment
% the following command:

%\beamerdefaultoverlayspecification{<+->}

\begin{frame}
\end{frame}
\end{document}



